\section{Conclusion}
\label{sec:conclusion}

Our modified Random Forests algorithm shows promising efficacy in classifying lattice phases in the critical region (in addition to other regions) with an efficiency best at a leaf size of 10000 with large estimators (our results present upto 160).
However, this comes at a cost of the algorithm being computationally expensive with a runtime of \( \approx \) 90-170 seconds.

This accuracy is comparable to the convolutional neural networks (CNN) implementation with an accuracy of 88-92\% \cite{2019}.

\section{Limitations and future prospects}

\subsection{One-dimensional Ising model}

We see that linear regression is bad for training our model for the 1D lattice.
If the dataset is small, Lasso predicts that our model accurately with a regularization \( \lambda \approx 10^{-2} - 10^{-4}\).
However, if the dataset is large, then both Lasso and Ridge regression predict an accurate model with regularization \( \lambda \approx 10^{-2} - 10^{-4}\).

For our experimentation with 2$^{nd}$ nearest neighbour interaction model, we see that Ridge regression works pretty well in predicting interactions in our lattice.

\subsection{Two-dimensional Ising model}

Our approach depended on a supervised learning model with labels indicating whether the lattice phase is ordered, disordered or critical.
This requires us to know beforehand the properties of the material under consideration in particular its critical point where the material transitions from a ferromagnetic to a paramagnetic phase, which means manually placing labels for ordered/disordered lattices to get a well-trained model.

We hope to work on using an unsupervised approach to get a ``generative'' rather than a ``discriminative'' model in the future.
These models are capable of generating a new critical phase lattice from the information of lattices we have in our data.
This is achieved using restricted Boltzmann machines and deep Boltzmann machines \cite{morningstar2017deep, salakhutdinov2009deep}.
