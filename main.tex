\documentclass[journal]{IEEEtran}
\usepackage[style=ieee]{biblatex}
\bibliography{references.bib}
\usepackage[T1]{fontenc}
\usepackage[scaled]{beramono}
\usepackage{amsmath}
\usepackage[none]{hyphenat}
\usepackage{tgschola}
\usepackage{tikz}
\usetikzlibrary{matrix}
\usepackage{authblk}
\usepackage{amssymb}
\usepackage{siunitx}
\usepackage{microtype}
\usepackage{fixmath}
\usepackage{graphicx}
\usepackage{minted}
\usepackage{hyperref}
\usepackage{cleveref}
\graphicspath{{./images/}}

% render vectors in bold
\renewcommand{\vec}[1]{\boldsymbol{#1}}
\newcommand{\Hst}{\textsc{H}-structure}

\setlength{\parskip}{3mm}
\setlength{\parindent}{0mm}
\usepackage{titlesec}
\titlespacing{\subsubsection}{0pt}{0cm}{0.1cm}

\begin{document}
\title{Applications of ML in investigating Ferromagnetic Transitions}

\author[1]{Gautameshwar S.\thanks{All authors have contributed equally.}}
\author[2]{Ashish Panigrahi}\affil[1,2]{School of Physical Sciences\\NISER, Bhubaneswar}%

\maketitle

\begin{abstract}
    We investigate the applications in machine learning in physical systems such as the one-dimensional and two-dimensional Ising spin models.
    With regards to the case of 1D, we observe that training a linear regression model for finding the coupling constant \( J \) leads to overfitting and variants of this regression model i.e. Lasso and Ridge regression yield better results with the featured advantage being that the model avoids overfitting.

    Implementation of the Metropolis algorithm was done to generate lattice data for the two-dimensional model.
    Our model was trained using supervised learning with a modified implementation of the Random Forests (RF) classifier scheme using a transductive approach.
    Comparison with the vanilla RF algorithm is seen with a clear score improvement in the case of our modified implementation.

    Future aspects of the work presents an opportunity to explore unsupervised schemes via deep Boltzmann machines.
\end{abstract}

\section{Introduction}

\subsection{Motivation}

The motivation behind our project is to investigate applications of machine learning in determining parameters in physical systems which are usually difficult to do so via a numerical approach.
In this project, we will be focusing on the one-dimensional and two-dimensional Ising spin models consisting of spin-1/2 atoms with either spin-\( \uparrow \) or spin-\( \downarrow \).

This project is divided into two parts:

\begin{enumerate}
    \item Finding the coupling constant \( J \) of the one-dimensional spin lattice.
    \item Classifying the phase orderness of a two-dimensional lattice depending on if the lattice temperature is beyond a critical point.
\end{enumerate}

\section{Theory}
\label{sec:theory} 

For any Ising spin model consisting of spin particles such as atoms/molecules, we consider a lattice (depending on the dimension of the system under consideration) with \( N \) spin particles, whose Hamiltonian is given by

\begin{equation}
    \vec{H} = \sum_{\langle i, j \rangle}^{N, N} J_{ij} \sigma_i \cdot \sigma_j
\end{equation}

where \( J_{ij} \) is the coupling strength (parameter quantifying the interaction between spin particles), \( \sigma_{i/j} \) is the spin of the particles (+1 for spin-\( \uparrow \) and -1 for spin-\( \downarrow \)).

Note that we ensure an interaction upto the first nearest neighbours i.e. immediate neighbours of a spin particle.
This is denoted by \( \langle i, j \rangle \).

\section{Implementation and Results}

\subsection{1D Ising Model}

The first problem we will attempt to solve using machine learning is how we can find the pattern of 2-body interaction of atoms in a given 1D spin lattice.
To get a flavour of what our objective is, consider the following:

You are given a spin configuration of a 50-atom long spin lattice in the form of a 1D array.
Each of these elements are either in spin up (+1) or spin down (-1).
Suppose we consider the atom at the 5th site and say this atom adds stability to our lattice if both the spins neighbouring to it has the same spin as itself.
By asserting such a statement, we have indirectly defined the energy of interaction between the $5^{th}$ atom and $6^{th}$, $7^{th}$ atoms.
This statement, translated into a physics equation, states:

\[ E_{system}=J_{5,6} \cdot \sigma_5\sigma_6 + J_{5,7} \cdot \sigma_5\sigma_7\]

where \(J_{5,6}, J_{5,7}\) are the strength of interaction between the atoms in the subscript.
If we extend the same argument to every single ith atom in the lattice interacting with every jth atom, we get the generalised 2-body interaction energy relation: \[E_{model}=\sum_{i,j=1}^{50}J_{ij}.\sigma_i\sigma_j\]

Now, if we are just given one lattice configuration with an energy label defined as above, there are countless possible \(J_{ij}\) matrices that are a valid solution to our above lattice.
If we manage to get all \(2^{50}\) unique lattice configurations, then there is analytically only one solution for our \(J_{ij}\).
But if we have some finite number of such 1D lattices with a energy label, it is possible to find the pattern in the nature of interaction of an atom with the other atoms of the lattice using regularised machine learning models.

\subsubsection{Data generation}

We assume a lattice at high temperature since temperature doesnt affect the J matrix.
A random assignment of spins in an 50 atom array along with the calculation of energy label based on the interaction matrix J defined by:

\begin{equation*}
    J_{ij} = \left\{
        \begin{array}{ll}
            -1 & \mbox{if } i = j \pm 1\\
            0 & \mbox{otherwise}
        \end{array}
    \right.
\end{equation*}

\begin{figure}[H]
    \centering
    \includegraphics[scale=0.6]{J-model.png}
    \caption{J matrix for nearest neighbour interaction}
    \label{fig:j-matrix}
\end{figure}

\subsubsection{Unregularised Linear Regression}

We implement the \emph{Least Square Regression} model available in scikit-learn \cite{alpaydin2020introduction}. The details provided below.
\begin{itemize}
    \item Parameter to train: \(J_{ij}\)
    \item Input variable: \(X_{ij}=\sigma_i\sigma_j\)
    \item Predicted Regression variable: \(E^n=X^n.J\), where $n$ is the $n^{th}$ lattice we input
    \item No of training lattices: N=10,000
    \item Loss function:
        \begin{equation*}
           \frac{1}{N} \sum_{n=1}^N(E_p^{(n)}-E_{label}^{(n)})^2
        \end{equation*}
    \item Score prediction method: Pearson's \(R^2\)
\end{itemize}

The results of linear regression turn out to be a perfect score for both training and testing data.
This is clearly too good to be true and this seems to be a case of overfitting.
The issue of overfitting is clearly seen in \cref{fig:overfit} if we plot the predicted $J$ matrix.

\begin{figure}[H]
    \centering
    \includegraphics[scale=0.4]{overfitting.png}
    \caption{Overfitted $J$ matrix}
    \label{fig:overfit}
\end{figure}

\begin{figure}[h!]
\begin{tikzpicture}
\matrix[matrix of nodes]{
    \includegraphics[scale=0.4]{lsq_n200.png} & \includegraphics[scale=0.4]{lsq_n1700.png}\\
    \includegraphics[scale=0.4]{lsq_n2500.png}\\
};
\end{tikzpicture}
\caption{Extent of overfitting for varying data size \( n \).}
\end{figure}


\begin{figure}[h!]
    \centering
    \includegraphics[scale=0.55]{lsq_Lattice_score-vs-datapts.png}
    \caption{Training and testing score for varying data size}
    \label{fig:lsqscore}
\end{figure}

The model overfits and produces perfect scores in both training and testing datasets after a certain size of input data because the statistical properties of our training and testing data becomes identical on large samples of lattices.
Thus overfitting gets away with it unattended!

\subsubsection{Regularised Lasso Regression}
We then implement the Lasso Regression model available in scikit-learn. The details provided below.
\begin{itemize}
    \item Parameter to train: \(J_{ij}\)
    \item Input variable: \(X_{ij}=\sigma_i\sigma_j\)
    \item Predicted Regression variable: \(E^n=X^n.J\), where n is the nth lattice we input
    \item Loss function: \(\lambda ||J||_1 + \frac{1}{N} \sum_{n=1}^N(E_p^{(n)}-E_{label}^{(n)})^2\)
    \item Score prediction method: Pearson's \(R^2\)
\end{itemize}

The implementation of Lasso regression gave significant improvement in the overfitting issue.
For smaller datasets, we found a ``window of sweet-spot'' for \(\lambda\) in which the model accurately predicted the lattice interactions without any issue of overfitting.
For larger datasets, even a small regularisation \(\lambda\) yields an accurate prediction of our interaction matrix.

\begin{figure}[h!]
    \centering
    \includegraphics[scale=0.55]{lasso_n-495.png}
    \caption{Training and testing score as a function of \( \lambda \) (\( n=495 \)).}
\end{figure}


\begin{figure}[h!]
    \centering
    \includegraphics[scale=0.55]{lasso_n-5000.png}
    \caption{Training and testing score as a function of \( \lambda \) (\( n=5000 \)).}
\end{figure}


\subsubsection{Ridge Regression}
Finally we implement the Ridge Regression model available in scikit-learn. The details provided below.
\begin{itemize}
    \item Parameter to train: \(J_{ij}\)
    \item Input variable: \(X_{ij}=\sigma_i\sigma_j\)
    \item Predicted Regression variable: \(E^n=X^n.J\), where n is the nth lattice we input
    \item Loss function: \(\lambda ||J||_2 + \frac{1}{N} \sum_{n=1}^N(E_p^{(n)}-E_{label}^{(n)})^2\)
    \item Score prediction method: Pearson's \(R^2\)
\end{itemize}

Unlike Lasso regression, Ridge did not offer any optimised regularisation \(\lambda\) for smaller datasets.
The model is seen to performed as poorly as LSQ.
However, on increasing the data size, our ridge performs better at predicting our interaction matrix without the issue of overfitting.

\begin{figure}[h!]
    \centering
    \includegraphics[scale=0.55]{ridge_n-493.png}
    \caption{Training and testing score as a function of \( \lambda \) (\( n=493 \)).}
\end{figure}

\begin{figure}[h!]
    \centering
    \includegraphics[scale=0.55]{ridge_n-3005.png}
    \caption{Training and testing score as a function of \( \lambda \) (\( n=3005 \)).}
\end{figure}



\subsubsection{Discussion on the results and further extension of our model}

We obtain contrastingly different results for Lasso and Ridge despite several similarities in them.
Both Ridge and Lasso don't like large values of \(||J||\) as it makes our data more sensitive to errors. They try to prefer $J$ with smaller ``slope'' than that with large one.
Thus, they regularise the overfitting issue efficiently.
But the optimisation of Lasso and ridge w.r.t LSQ is different.
It can be mathematically shown that:

\[J_{ridge}=\frac{J_{lsq}}{1+\lambda}\]

Thus, Ridge always scales the $J$ by the regularisation parameter.
Larger the \(\lambda\), more flat our ``slope'' becomes.
But even for large \(\lambda\), $J$ only approaches towards 0 asymptotically (close but never equal to 0).

\begin{figure}[H]
    \centering
    \includegraphics[scale=0.5]{ridge-vs-lasso.png}
    \caption{Visual schematic comparing Ridge and Lasso regression effects.}
\end{figure}

Lasso, on the other hand doesn't care about asymptotically reducing $J$.
As we increase \(\lambda\), the gap between the \(J_{lsq}\) and \(J_{lasso}\) where $J=0$ widens.
This results in more and more unimportant parts of our data being set to 0 (which we don't see in ridge).
Thus, for some specific \(\lambda\), we have exactly the unnecessary data set to 0, and the prominent parts of our ``slope'' $J$ intact.
This is the reason we see sweet spots that perfectly regularise by making the unnecessary overfitting to 0 and only retain the original model.
This is phenomenon is also referred to as ``sparse'' regression.

The Ridge model works at finding atom interaction beyond neighbours. We tested this by generating a data of lattices and labelling them with energy now defined by:

\begin{equation*}
    J_{ij} = \left\{
        \begin{array}{ll}
            -1 & \mbox{if } i = j \pm 1\\
            -0.5 & \mbox{if } i = j \pm 2\\
            0 & \mbox{otherwise}
        \end{array}
    \right.
\end{equation*}

\begin{figure}[H]
    \centering
    \includegraphics[scale=0.55]{J_2atoms_2.png}
    \caption{J matrix defined for 2$^{nd}$ nearest interaction lattice}
\end{figure}

\begin{figure}[H]
    \centering
    \includegraphics[scale=0.55]{J_2atoms_2_ridge.png}
    \caption{J matrix predicted by Ridge}
\end{figure}

\begin{itemize}
    \item n: 5000
    \item train-test: 3500 - 1500
    \item Training score: 99.99/100
    \item Testing score: 99.99 /100
\end{itemize}

\subsection{2D Ising Training Data}
With our goal as to predict the ordered/disordered state of our lattice, we are in need of generating lots of Ising lattice data for each of the temperature steps about the critical temperature \(T_c\).
A 2D Ising lattice maintained at a temperature $T$ constantly fights between stabilising itself by lowering its energy and allowing itself to excite to a higher energy configuration due to the thermal agitation existing.
This is the result of the entropy of our lattice being minimised by establishing a statistical equilibrium between itself and the temperature reservoir in which it is existing.
The standard algorithm to simulate this statistical phenomenon is the Metropolis algorithm \cite{zhaodetermining}.

\subsubsection{The Metropolis Algorithm}
The goal of this algorithm is to find an equilibrium state for the two-dimensional spin lattice system at a particular temperature dictated by \(\beta\). It is achieved as follows:
\begin{enumerate}
    \item Begin with a random state \(\mu\) for the lattice configuration.
    \item Pick any random lattice site and flip the sign of the spin. We call this new state \(\nu\). What is the probability $P(\mu \rightarrow \nu)$ that we will accept this new state?
    \item If \(E_{\mu}>E_{\nu}\), then we set $P_{\nu \rightarrow \mu} = 1$ and thus by using the detailed balance equation we get $P_{\nu \rightarrow \mu} =e^{-\beta(E_{\nu}-E_{\mu})}$
    \item If $E_{\nu}>E_{\mu}$, then we set $P_{\mu \rightarrow \nu} = 1$ (i.e) with full probability, the spins gets flipped to a more stable configuration.
    \item Iterate step (1)-(4) as many times as required. Eventually we end up with an equilibrium state for large enough iterations
\end{enumerate}

The convergence for a solution is guaranteed for large enough iterations. It can be statistically shown that the relative error for the above Monte-Carlo goes as $\Delta E/E \approx 1/\sqrt{N}$ for large iterations $N$.

Our implementation of the Metropolis algorithm is done by defining a class 2D Ising model that has the following parameters and objects:

Input parameters while defining a class object of 2D ising model:
\begin{enumerate}
    \item \texttt{d}: Dimension of our 2D lattice. \texttt{dtype=int}
    \item \texttt{a}: Expected ratio of down spins to up spins in our randomly generated lattice
    \item \texttt{mfield}: Option to add the presence of an external magnetic field in our existing lattice.
\end{enumerate}

Objects in our 2D ising model class:
\begin{enumerate}
    \item \texttt{lattice}: Our randomly generated lattice (with 0 padded on the edge) on defining the 2D ising model object.
    \item \texttt{energy}: Computed energy of self.lattice
    \item \texttt{spin}: Computed net spin of self.lattice
\end{enumerate}

Functions in our 2D ising model class:
\begin{enumerate}
    \item \texttt{stable\_energy}: Function to compute the most stable energy configuration
    \item \texttt{get\_energy}: Function to compute the energy of \texttt{self.lattice}
    \item \texttt{flip\_check}: Inputs x,y co-ordinates, outputs the energy before and after flipping the spin at (x,y)
    \item \texttt{flip}: Inputs x,y co-ordinates, flips the spin at (x,y) of \texttt{self.lattice}
\end{enumerate}

We performed a numba implementation of our metropolis algorithm as it is extremely slow to perform iterations of order ~ 100,000 in a standard python executer. \\
Runtime of generating one 50x50 lattice after 100,000 metro iterations using a python executer: ~ 47 secs\\
Runtime of generating one 50x50 lattice after 100,000 metro iterations using Numba: ~ 0.18 secs

The data is organised as 10,000 Metro-generated lattices for one single temperature $T$. We have 20 such equally spaced temperatures in the range of $T$: 0.25-4 units.

Iteration plots of our Metropolis algorithm shows that our algorithm exhibits convergence for lower temperatures and fluctuates and becomes noisy with increase in temperature.
This is solely due to the fact that our system has an energy range for its likelihood to exist.
These energy states almost have equal likelihood due to high $T$, and this does not correspond to the fact that our algorithm misbehaves at those cases.
Spins usually do not converge since there are theoretically many spin configurations that might correspond to the same energy state.

\begin{figure}[h!]
    \centering
    \includegraphics[scale=0.6]{Figure_28.png}
    \caption{Ordered lattice generation.\protect\footnotemark}
\end{figure}

\footnotetext{\texttt{\#iterations} along the x-axis denotes ``Number of iterations''.}

\begin{figure}[h!]
    \centering
    \includegraphics[scale=0.6]{Figure_10.png}
    \caption{Critical lattice generation.}
\end{figure}

\begin{figure}[h!]
    \centering
    \includegraphics[scale=0.6]{Figure_1.png}
    \caption{Disordered lattice generation.}
\end{figure}


\subsection{2D Phase Transition Training}

\subsubsection{Labelling and implementing Random Forests Algorithm}

Now that we have generated our data, our objective now is to tell apart the ordered lattices from the disordered. The first task to achieve is to analytically find the critical temperature $T_c$ after which our lattices become disordered. This is given by \cite{onsager1944crystal}:

\[T_c=\frac{J}{\log(1+\sqrt{2})}\]

where $J$ is the strength of interaction of our atoms with its neighbours. With $J$=2 in our case, we have $T_c=2.26$.

\begin{figure}[h!]
    \centering
    \includegraphics[scale=0.5]{cvst.png}
    \caption{Heat capacity vs temperature of lattice}
\end{figure}

\begin{figure}[h!]
    \centering
    \includegraphics[scale=0.5]{evst.png}
    \caption{Energy vs temperature of lattice}
\end{figure}

Thus, all lattices above $T=2.26$ is labelled 0 (as disordered lattice) and the ones below are labelled 1 (as ordered lattice).
We further split our lattice into ordered, critical and disordered and proceed to only use the ordered and disordered lattices for training our model.
We reserve the set of critical lattices for verifying our models performance in critical region as it is expected to struggle to classify lattices in that region.
A better critical score means our model has learnt efficiently just from the ordered and disordered lattices on how to classify our lattices in the critical regime, along with the other regimes.

We implement a Random Forests (RF) classifier from scikit-learn as our training model for the above method due to the below reasons:

\begin{itemize}
    \item Our lattices are Monte-Carlo generated where our predictions are usually correlated. A spin flip in the initial stage of the Metropolis algorithm can result is an entirely new lattice configuration in the end. Thus weak classifiers such as decision trees and averaging several such models reduces the risk of choosing the wrong model (lowering the overall variance and bias).
    \item Ensemble methods such as RF reduces the risk of our model to rely on some greedy assumption or local search that may get stuck in a local extrema (quite common in Monte-Carlo data such as ours) and thus, it generates a better predicting model.
\end{itemize}

The Random Forests algorithm is implemented as follows:
\begin{itemize}
    \item Input variable: \(X=[x^i],x^i\in {X_{ordered}\bigcup X_{disordered}}\)
    \item Classifiers used: Decision Trees
    \item Hyperparameters used in tuning our optimal RF:
        \item \texttt{n\_estimators}: No. of decision trees in our forest.
        \item \texttt{min\_sample\_split}: No. of samples needed to split an internal node.
    \item Random sampling method: Bootstrap sampling
    \item Additional optimisation: Warm start=True. This ensures that you add estimators without fitting a whole new forest every time.
    \item Score prediction method for ordered and disordered samples: OOB estimators
    \item Score prediction method for critical samples: mean accuracy score (inbuilt score function in our training model)
\end{itemize}

The hyperparameter \texttt{min\_sample\_split} has two extremes: coarse and fine.
Increasing the above hyperparameter makes our forest much coarser.

The results obtained by Mehta \emph{et al.} \cite{2019} shows a poor performance in the critical samples though almost perfect for ordered and disordered samples. In the critical samples, vanilla RF gave a best accuracy of 69.2\% for coarse trees (leaf size: 10,000) compared to a accuracy score of 83.1\% for fine trees (leaf size: 2) at 100 estimators each.

\begin{figure}[h!]
    \centering
    \includegraphics[scale=0.45]{mehta.png}
    \caption{Scores for various phases obtained in \cite{2019}.}
\end{figure}


\subsubsection{Improving the algorithm: Transductive Approach}
Given that we see the fact our model struggles in the critical region, we expect a deeper learning in the critical region will help our model get over the poor performance.
We implemented this idea into our random by introducing a transduction in our model by weighing the lattices closer to the critical regime more than the ones far from it.

\begin{figure}[H]
    \centering
    \includegraphics[scale=0.5]{region.png}
    \caption{Schematic of our weighting of the lattices in our modified implementation.}
\end{figure}

Since scikit-learn does not have the option to add sample weight as an input, we performed data cloning so that our random forest encounters these samples more and trains more on them. This simple hack will aid our algorithm to pick features more prominent in the critical region and by this emphasis, it trains deep in those features. It also lowers the priority of features present in completely ordered/disordered lattices since they don’t play as important of a role as the lattices that are in the neighbourhood of the critical point.

Results of our modified Random Forests are as follows,

\begin{figure}[h!]
    \centering
    \includegraphics[scale=0.45]{train-score.png}
    \caption{Scores for training datasets obtained as a function of \( N_{\text{estimators}} \).}
\end{figure}

\begin{figure}[h!]
    \centering
    \includegraphics[scale=0.45]{test-score.png}
    \caption{Scores for testing datasets obtained as a function of \( N_{\text{estimators}} \).}
\end{figure}

\begin{figure}[h!]
    \centering
    \includegraphics[scale=0.45]{critical-score.png}
    \caption{Scores for the critical phases obtained as a function of \( N_{\text{estimators}} \).}
\end{figure}

The results show an increase in the accuracy of the classification of critical samples compared to vanilla RF. Interestingly, by emphasising on samples closer to critical temperature $T_c$, our altered version of RF works better
for coarse trees (92.0\%) than fine ones (88.75\%). A reason for this might be due to the fact that our implementation of RF identified many features in the critical samples that made it effect in larger leaf sizes than smaller ones. We note that our model still retains its efficiency in classifying the ordered and disordered samples as the OOB scores for these samples are still good.

\section{Conclusion}
\label{sec:conclusion}

Our modified Random Forests algorithm shows promising efficacy in classifying lattice phases in the critical region (in addition to other regions) with an efficiency best at a leaf size of 10000 with large estimators (our results present upto 160).
However, this comes at a cost of the algorithm being computationally expensive with a runtime of \( \approx \) 90-170 seconds.

This accuracy is comparable to the convolutional neural networks (CNN) implementation with an accuracy of 88-92\% \cite{2019}.

\section{Limitations and future prospects}

\subsection{One-dimensional Ising model}

We see that linear regression is bad for training our model for the 1D lattice.
If the dataset is small, Lasso predicts that our model accurately with a regularization \( \lambda \approx 10^{-2} - 10^{-4}\).
However, if the dataset is large, then both Lasso and Ridge regression predict an accurate model with regularization \( \lambda \approx 10^{-2} - 10^{-4}\).

For our experimentation with 2$^{nd}$ nearest neighbour interaction model, we see that Ridge regression works pretty well in predicting interactions in our lattice.

\subsection{Two-dimensional Ising model}

Our approach depended on a supervised learning model with labels indicating whether the lattice phase is ordered, disordered or critical.
This requires us to know beforehand the properties of the material under consideration in particular its critical point where the material transitions from a ferromagnetic to a paramagnetic phase, which means manually placing labels for ordered/disordered lattices to get a well-trained model.

We hope to work on using an unsupervised approach to get a ``generative'' rather than a ``discriminative'' model in the future.
These models are capable of generating a new critical phase lattice from the information of lattices we have in our data.
This is achieved using restricted Boltzmann machines and deep Boltzmann machines \cite{morningstar2017deep, salakhutdinov2009deep}.


\printbibliography

\appendix
\section{Code}

\subsection{Least Square Regression}

\onecolumn
\begin{minted}{python}
import numpy as np
import matplotlib.pyplot as plt
from sklearn import linear_model

# function to return the energy of our lattice
def energy(l):
    e = 0
    j = 1
    # our energy assumes periodic boundary conditions
    for i in range(l.size - 1):
        e += -j * l[i] * (l[i - 1] + l[i + 1])

    e += -j * l[l.size - 1] * (l[l.size - 2] + l[0])
    return e


# length of our lattice
d = 50
for n in [250, 800, 1700, 2000, 4000, 5000]:
    # defining our lattice
    l = np.random.choice([-1, 1], size=(n, d))

    # array to store the labels
    label = np.zeros(n)

    # creating the labels
    for i in range(n):
        label[i] = energy(l[i])

    # partition of our training n testing sets
    m = int(7 * n / 10)

    train = l[:m]
    test = l[m:]

    train_label = label[:m]
    test_label = label[m:]

    # data generated!!

    # setting up our ML algos
    leastsq = linear_model.LinearRegression()

    # needed for training the data: computing the outer product of S_i
    train_states = np.einsum("...i,...j->...ij", train, train)
    shape = train_states.shape

    # each row is a reshaped from a 5x5 matrix of S_i*S_j
    train_states = train_states.reshape((shape[0], shape[1] * shape[2]))

    # needed for training the data: computing the outer product of S_i
    test_states = np.einsum("...i,...j->...ij", test, test)
    shape = test_states.shape

    # each row is a reshaped from a 5x5 matrix of S_i*S_j
    test_states = test_states.reshape((shape[0], shape[1] * shape[2]))

    leastsq.fit(train_states, train_label)
    J = leastsq.coef_.reshape((shape[1], shape[2]))

    mark1 = leastsq.score(train_states, train_label) * 100
    mark2 = leastsq.score(test_states, test_label) * 100

    print("n: %s" % n, "train-test: ", m, "-", n - m)
    print("Training score: %s /100" % mark1)
    print("Testing score: %s /100" % mark2, "\n")

    plt.figure()
    plt.title("n=%s" % n)
    c = plt.imshow(
        J, vmin=-1, vmax=1, cmap="Spectral_r", interpolation="nearest", origin="lower"
    )
    plt.colorbar(c)
\end{minted}

\subsection{1D training using Lasso regression}

Here, we adopt the same \texttt{energy()} function as used in least square regression but with training done via Lasso regression.

\begin{minted}{python}
import numpy as np
import matplotlib.pyplot as plt
from sklearn import linear_model

# function to return the energy of our lattice
def energy(l):
    e = 0
    j = 1
    # our energy assumes periodic boundary conditions
    for i in range(l.size - 1):

        e += -j * l[i] * (l[i - 1] + l[i + 1])

    e += -j * l[l.size - 1] * (l[l.size - 2] + l[0])

    return e

# length of our lattice
d = 50
# no of 1d ising models
narr = np.linspace(100, 5000, 100, dtype="int")

# obtain score relation with varying number of datapoints
for n in narr:
    score_tr = []
    score_te = []

    # defining our lattice
    l = np.random.choice([-1, 1], size=(n, d))

    # array to store the labels
    label = np.zeros(n)

    # creating the labels
    for i in range(n):
        label[i] = energy(l[i])

    # partition of our training n testing sets
    m = int(7 * n / 10)

    train = l[:m]
    test = l[m:]

    train_label = label[:m]
    test_label = label[m:]

    # data generated!!

    # setting up our ML algos
    lasso = linear_model.Lasso(tol=0.01)

    # needed for training the data: computing the outer product of S_i
    train_states = np.einsum("...i,...j->...ij", train, train)
    shape = train_states.shape

    # each row is a reshaped from a 5x5 matrix of S_i*S_j
    train_states = train_states.reshape((shape[0], shape[1] * shape[2]))

    # needed for training the data: computing the outer product of S_i
    test_states = np.einsum("...i,...j->...ij", test, test)
    shape = test_states.shape

    # each row is a reshaped from a 5x5 matrix of S_i*S_j
    test_states = test_states.reshape((shape[0], shape[1] * shape[2]))

    l = np.logspace(-5, 6, 100)
    l = np.round(l, 5)
    i = 0

    # obtain score relatn with varying regularisation parameter lambda
    for a in l:
        i += 1

        # training our dataset
        lasso.set_params(alpha=a)
        lasso.fit(train_states, train_label)
        J = lasso.coef_.reshape((shape[1], shape[2]))

        mark1 = lasso.score(train_states, train_label) * 100
        mark2 = lasso.score(test_states, test_label) * 100

        score_tr.append(mark1)
        score_te.append(mark2)

    plt.figure()
    plt.grid()
    plt.title(r"Variation of train and test score. n: %s" % n)
    plt.xlabel(r"log($\lambda$)")
    plt.ylabel(r"$R^{2}$ (in %)")
    plt.plot(np.log(l), score_tr, label="Training Score")
    plt.plot(np.log(l), score_te, label="Testing Score")
    plt.legend()
    plt.savefig(r"%s.png" % n)
\end{minted}

\subsection{1D training using Ridge regression}

\begin{minted}{python}
import numpy as np
import matplotlib.pyplot as plt
from sklearn import linear_model

# function to return the energy of our lattice
def energy(l):
    e = 0
    j = 1
    # our energy assumes periodic boundary conditions
    for i in range(l.size - 1):

        e += -j * l[i] * (l[i - 1] + l[i + 1])

    e += -j * l[l.size - 1] * (l[l.size - 2] + l[0])

    return e


# length of our lattice
d = 50
# no of 1d ising models
n = 10000
score_tr = []
score_te = []

# defining our lattice
l = np.random.choice([-1, 1], size=(n, d))

# array to store the labels
label = np.zeros(n)

# creating the labels
for i in range(n):
    label[i] = energy(l[i])

# partition of our training n testing sets
m = int(7 * n / 10)

train = l[:m]
test = l[m:]

train_label = label[:m]
test_label = label[m:]

# data generated!!

# setting up our ML algos
leastsq = linear_model.LinearRegression()
ridge = linear_model.Ridge(tol=0.01)
lasso = linear_model.Lasso()

# needed for training the data: computing the outer product of S_i
train_states = np.einsum("...i,...j->...ij", train, train)
shape = train_states.shape

# each row is a reshaped from a 5x5 matrix of S_i*S_j
train_states = train_states.reshape((shape[0], shape[1] * shape[2]))

# needed for training the data: computing the outer product of S_i
test_states = np.einsum("...i,...j->...ij", test, test)
shape = test_states.shape

# each row is a reshaped from a 5x5 matrix of S_i*S_j
test_states = test_states.reshape((shape[0], shape[1] * shape[2]))

l = np.logspace(-5, 6, 100)
l = np.round(l, 5)
i = 0
for a in l:
    i += 1

    # training our dataset
    ridge.set_params(alpha=a)
    ridge.fit(train_states, train_label)
    J = ridge.coef_.reshape((shape[1], shape[2]))

    mark1 = ridge.score(train_states, train_label) * 100
    mark2 = ridge.score(test_states, test_label) * 100

    score_tr.append(mark1)
    score_te.append(mark2)

    plt.figure()
    plt.title(r"$\lambda$=%s" % a)
    c = plt.imshow(
        J, vmin=-1, vmax=1, cmap="Spectral_r", interpolation="nearest", origin="lower"
    )
    plt.colorbar(c)
    plt.savefig(r"%s.png" % i)

plt.figure()
plt.grid()
plt.title(r"Variation of train and test score. n: %s" % n)
plt.xlabel(r"$\lambda$")
plt.ylabel(r"$R^{2}$ (in %)")
plt.plot(np.log(l), score_tr, label="Training Score")
plt.plot(np.log(l), score_te, label="Testing Score")
plt.legend()
plt.show()
\end{minted}

\subsection{Implementation of 2nd nearest neighbour interaction for 1D Ising model}

This is only a snippet of code for implementing one-dimensional model where 2$^{nd}$ nearest neighbour interaction.

\begin{minted}{python}

# defining energy function for next to next atom interactions

# function to take care of periodic index of our lattice
def pbc(i, d):
    if i < d:
        return i

    else:
        return d - i


# function to return the energy of our lattice
def energy(l, d):
    e = 0
    # center is the atom, left n right r interactions with neighbours
    j = [-0.5, -1, 0, -1, -0.5]

    # our energy assumes periodic boundary conditions
    for i in range(l.size):

        e += l[i] * (
            j[pbc(0, d)] * l[pbc(i - 2, d)]
            + j[pbc(1, d)] * l[pbc(i - 1, d)]
            + j[pbc(2, d)] * l[pbc(i, d)]
            + j[pbc(3, d)] * l[pbc(i + 1, d)]
            + j[pbc(4, d)] * l[pbc(i + 2, d)]
        )

    return e
\end{minted}

\subsection{Generation of 2D Ising data using Metropolis}

\begin{minted}{python}
import numpy as np
import numba
import time

# define an ising model class to simplify calculations...
class ising_model_2D:
    # inputs our lattice, keeps the data of our lattice's energy and spin tagged
    def __init__(self, d, a, mfield):
        # lattice with a:(1-a) down-up
        temp = np.random.random((d, d))

        temp[temp >= a] = 1
        temp[temp < a] = -1

        # create 0 in the border (just to make neighbour atom calculations easier)
        temp = np.pad(array=temp, pad_width=1, mode="constant")
        self.lattice = temp.copy()
        del temp

        self.b = mfield

        # initialise the energy n net spin of our lattice
        self.energy = self.get_energy()  # use it to get monte carlo data.
        self.spin = np.sum(self.lattice)

    # to find the stable energy of our lattice
    def stable_energy(self):
        # the most stable energy(alligned spins))
        n = np.ones((d, d))
        n = np.pad(array=n, pad_width=1, mode="constant")

        e = 0
        # look at the i,jth atom.
        for i in range(1, d + 1):
            for j in range(1, d + 1):
                # see the neighbouring atoms(up,down,left,right)
                g = np.array([[0, 1, 0], [1, 0, 1], [0, 1, 0]])

                # calculate the sigma_ij*sigma_i'j' around our atom at ij
                m = -n[i, j] * g * n[i - 1 : i + 2, j - 1 : j + 2]

                # sum over all the neighbour sigmas
                e += np.sum(m)

        return e

    # fn to find the energy of our lattice with B
    def get_energy(self):
        e = 0
        # look at the i,jth atom.
        for i in range(1, d + 1):  # correction: d->d+1..!
            for j in range(1, d + 1):
                # see the neighbouring atoms(up,down,left,right)
                g = np.array([[0, 1, 0], [1, 0, 1], [0, 1, 0]])

                # calculate the sigma_ij*sigma_i'j' around our atom at ij
                m = -self.lattice[i, j] * g * self.lattice[i - 1 : i + 2, j - 1 : j + 2]

                # sum over all the neighbour sigmas
                e += np.sum(m) - self.b * self.lattice[i, j]

        return e

    # fn to calculate the energy of lattice on flipping spin at (x,y)
    def flipcheck(self, x, y):
        # the energy change comes only from the neighbours of the spin we flipped at (x,y)
        g = np.array([[0, 1, 0], [1, 0, 1], [0, 1, 0]])

        # calculate the sigma_ij*sigma_i'j' around our atom at ij
        m = -self.lattice[x, y] * g * self.lattice[x - 1 : x + 2, y - 1 : y + 2]

        # sum over all the neighbour sigmas and the effect of B
        e1 = np.sum(m) - self.lattice[x, y] * self.b

        # flip the (x,y)th spin
        self.lattice[x, y] *= -1

        # print('in flip fn', self.lattice)

        # calculate the sigma_ij*sigma_i'j' around our atom at ij
        m = -self.lattice[x, y] * g * self.lattice[x - 1 : x + 2, y - 1 : y + 2]

        # sum over all the neighbour sigmas and the effect of B
        e2 = np.sum(m) - self.lattice[x, y] * self.b

        # flip bacc the (x,y)th spin
        self.lattice[x, y] *= -1
        # print('after flip fn', self.lattice)
        return e1, e2

    # fn to flip the (x,y)th spin
    def flip(self, x, y):
        # spin at (x,y) flippd
        self.lattice[x, y] *= -1

        # update the energy of our system after the iterations...
        e1, e2 = self.flipcheck(x, y)

        # the energy difference is double counted for all the neighbouring bonds
        self.energy += 2 * (e1 - e2)

        # update spin
        self.spin += 2 * self.lattice[x, y]


# dimension of our array
d = 50
# optional addition of external magnetic field B
b = 0.0

# metropolis algorithm
@numba.njit("f8[:,:](f8[:,:], f8, i8)", nogil=True)
def metro(lattt, beta, reps):
    lat = lattt.copy()
    for i in range(reps):

        # random pick of our point in lattice
        x, y = np.random.randint(1, d), np.random.randint(1, d)

        # energies of our two systems
        # e1, e2 = flipcheck(lat, x, y)

        # manual spinflip check
        g = np.array([[0, 1, 0], [1, 0, 1], [0, 1, 0]])
        m = -lat[x, y] * g * lat[x - 1 : x + 2, y - 1 : y + 2]
        e1 = np.sum(m) - lat[x, y] * b
        lat[x, y] *= -1
        m = -lat[x, y] * g * lat[x - 1 : x + 2, y - 1 : y + 2]
        e2 = np.sum(m) - lat[x, y] * b
        lat[x, y] *= -1

        # if energy increase only keep with a probability(more energy increase, lesser the prob)
        if e2 > e1:
            # probability of flipping if unfavourable
            p = np.exp(-beta * (e2 - e1))

            if np.random.random() < p:

                lat[x, y] *= -1

        # flip the lattice spin if energy decreases
        else:
            lat[x, y] *= -1

    return lat


# temperature fineness
h = 20
t = np.linspace(0.25, 4, h)

master = []
for T in t:

    data = []
    print("T: %s" % T)
    print("Data generated:")

    # no of datasets in each temperature
    n = 10000
    mat = time.time()
    for i in range(n):
        print(i)
        start = time.time()
        lat = ising_model_2D(d, 0.5, b)

        lat.lattice = metro(lat.lattice, 1 / T, 60000)
        lat.energy = lat.get_energy()
        lat.spin = np.sum(lat.lattice)

        data.append(lat)
        # plt.imshow(lat.lattice, cmap="Greys_r", interpolation="nearest", origin="lower")
        print("Execution time: ", time.time() - start)
    # save data in a file! Gets destroyed every iteration!!!
    data = np.array(data)
    master.append(data)
    print("Total time for beta: ", time.time() - mat, "\n")

master = np.array(master)
np.save("2d_ising_data_2", master, allow_pickle=True)
np.savetxt("Temperature.csv", t, delimiter=",")
\end{minted}

\subsection{Finding average energy and heat capacity of the generated 2D lattices}

\begin{minted}{python}

# 2d ising data load
data = np.load("2d_ising_data_2.npy", allow_pickle=true)

# temperature label load
t = np.genfromtxt("temperature.csv", delimiter=",")

# to see the statistical results of our generated data
e, spin = np.zeros(20), np.zeros(20)

print("size of temperature data: ", t.size, "\n")
print(
    "size of 2d ising model data: ",
    data.shape,
    "\n no.of temperature samples taken: ",
    data.shape[0],
    "\n no.of lattice per temperature sample: ",
    data.shape[1],
)

# plot the energy specific heat graffs
h = t.size
n = data.shape[1]

for i in range(h):
    avg_e = 0
    avg_spin = 0

    for j in range(n):
        avg_e += data[i][j].energy
        avg_spin += data[i][j].spin
    avg_e /= n
    avg_spin /= n

    e[i] = avg_e
    spin[i] = avg_spin


plt.figure()
plt.grid()
plt.title("avg. energy vs temperature")
plt.xlabel("t (units)")
plt.ylabel("e (units)")
plt.plot(t, e)
plt.plot(t, e, "k .")
plt.show()

c = np.zeros(h)
r = 1

for i in range(r, h - r):
    c[i] = (e[i + r] - e[i - r]) / (t[i + r] - t[i - r])

plt.figure()
plt.grid()
plt.title("heat capacity of lattice vs temperature")
plt.xlabel("t (units)")
plt.ylabel("de/dt (units)")
plt.plot(t[r : h - r], c[r : h - r])
plt.plot(t[r : h - r], c[r : h - r], "k .")
plt.show()
\end{minted}

\subsection{Training 2D Ising data using Random Forests}

\begin{minted}{python}
import numpy as np
import matplotlib.pyplot as plt
import time

# 2d ising data load
data = np.load("2d_ising_data_2.npy", allow_pickle=True)

# temperature label load
t = np.genfromtxt("temperature.csv", delimiter=",")

for i in range(20):
    for j in range(10000):
        data[i][j].lattice = data[i][j].lattice[1:-1, 1:-1].flatten()

data_ordered = data[:8]
data_critical = data[8:14]
data_disordered = data[14:]

# our training data
X_ordered = []
for i in range(len(data_ordered[:])):
    # lattice very close to critical temperature
    if i == len(data_ordered[:]) - 1:
        for j in range(data_ordered[0].size):
            X_ordered.append(data_ordered[i][j].lattice)
            X_ordered.append(data_ordered[i][j].lattice)
            X_ordered.append(data_ordered[i][j].lattice)
    # lattice bit near to critical temperature
    elif i == len(data_ordered[:]) - 2:
        for j in range(data_ordered[0].size):
            X_ordered.append(data_ordered[i][j].lattice)
            X_ordered.append(data_ordered[i][j].lattice)
    # lattice bit far from critical temperature
    else:
        for j in range(data_ordered[0].size):
            X_ordered.append(data_ordered[i][j].lattice)

X_disordered = []
for i in range(len(data_disordered[:])):
    # lattice very close to critical temperature
    if i == len(data_disordered[:]) - 1:
        for j in range(data_disordered[0].size):
            X_disordered.append(data_disordered[i][j].lattice)
            X_disordered.append(data_disordered[i][j].lattice)
            X_disordered.append(data_disordered[i][j].lattice)
    # lattice bit near to critical temperature
    elif i == len(data_disordered[:]) - 2:
        for j in range(data_disordered[0].size):
            X_disordered.append(data_disordered[i][j].lattice)
            X_disordered.append(data_disordered[i][j].lattice)
    # lattice bit far from critical temperature
    else:
        for j in range(data_disordered[0].size):
            X_disordered.append(data_disordered[i][j].lattice)

X_critical = []
for i in range(len(data_critical[:])):
    for j in range(data_critical[0].size):
        X_critical.append(data_critical[i][j].lattice)

del data, data_ordered, data_disordered, data_critical

# data labels for our data
y_ordered, y_critical, y_disordered = [], [], []

for i in range(len(X_ordered)):
    y_ordered.append(1)

for i in range(len(X_disordered)):
    y_disordered.append(0)

for k in range(len(X_critical) // 2):
    y_critical.append(1)

for k in range(len(X_critical) - len(X_critical) // 2):
    y_critical.append(0)

X = np.concatenate((X_ordered, X_disordered))
y = np.concatenate((y_ordered, y_disordered))

from sklearn.model_selection import train_test_split

train_to_test_ratio = 0.8

X_train, X_test, y_train, y_test = train_test_split(
    X, y, train_size=train_to_test_ratio, test_size=1.0 - train_to_test_ratio
)

print("X_train shape:", X_train.shape)
print("Y_train shape:", y_train.shape)
print()
print(X_train.shape[0], "train samples")
print(len(X_critical), "critical samples")
print(X_test.shape[0], "test samples")

# Training model
from sklearn.ensemble import RandomForestClassifier

min_estimators = 10
max_estimators = 101
classifer = RandomForestClassifier

n_estimator_range = np.arange(min_estimators, max_estimators, 10)
leaf_size_list = [10, 2500, 5000, 7500, 10000]

m = len(n_estimator_range)
n = len(leaf_size_list)

RFC_OOB_accuracy = np.zeros((n, m))
RFC_train_accuracy = np.zeros((n, m))
RFC_test_accuracy = np.zeros((n, m))
RFC_critical_accuracy = np.zeros((n, m))
run_time = np.zeros((n, m))

print_flag = True

for i, leaf_size in enumerate(leaf_size_list):
    # Define Random Forest Classifier
    myRF_clf = classifer(
        n_estimators=min_estimators,
        max_depth=None,
        min_samples_split=leaf_size,  # minimum number of sample per leaf
        oob_score=True,
        random_state=0,
        warm_start=True,  # this ensures that you add estimators without retraining everything
    )
    for j, n_estimator in enumerate(n_estimator_range):

        print("n_estimators: %i, leaf_size: %i" % (n_estimator, leaf_size))

        start_time = time.time()
        myRF_clf.set_params(n_estimators=n_estimator)
        myRF_clf.fit(X_train, y_train)
        run_time[i, j] = time.time() - start_time

        # check accuracy
        RFC_train_accuracy[i, j] = myRF_clf.score(X_train, y_train)
        RFC_OOB_accuracy[i, j] = myRF_clf.oob_score_
        RFC_test_accuracy[i, j] = myRF_clf.score(X_test, y_test)
        RFC_critical_accuracy[i, j] = myRF_clf.score(X_critical, y_critical)
        if print_flag:
            result = (
                run_time[i, j],
                RFC_train_accuracy[i, j],
                RFC_OOB_accuracy[i, j],
                RFC_test_accuracy[i, j],
                RFC_critical_accuracy[i, j],
            )
            print(
                "{0:<15}{1:<15}{2:<15}{3:<15}{4:<15}".format(
                    "time (s)",
                    "train score",
                    "OOB estimate",
                    "test score",
                    "critical score",
                )
            )
            print("{0:<15.4f}{1:<15.4f}{2:<15.4f}{3:<15.4f}{4:<15.4f}".format(*result))

plt.figure()
plt.title("Training Score")
plt.grid()
for i in range(n):
    plt.plot(
        n_estimator_range,
        RFC_train_accuracy[i],
        "--",
        label="Samples each node-%s" % leaf_size_list[i],
    )
    plt.plot(n_estimator_range, RFC_train_accuracy[i], ".k")


plt.xlabel("$N_\mathrm{estimators}$")
plt.ylabel("Accuracy")
lgd = plt.legend(bbox_to_anchor=(1.05, 1), loc=2, borderaxespad=0.0)

plt.show()

plt.figure()
plt.title("Testing Score")
plt.grid()
for i in range(n):
    plt.plot(
        n_estimator_range,
        RFC_test_accuracy[i],
        "--",
        label="Samples each node-%s" % leaf_size_list[i],
    )
    plt.plot(n_estimator_range, RFC_test_accuracy[i], ".k")

plt.xlabel("$N_\mathrm{estimators}$")
plt.ylabel("Accuracy")
lgd = plt.legend(bbox_to_anchor=(1.05, 1), loc=2, borderaxespad=0.0)

plt.show()

plt.figure()
plt.title("Critical Score")
plt.grid()
for i in range(n):
    plt.plot(
        n_estimator_range,
        RFC_critical_accuracy[i],
        "--",
        label="Samples each node-%s" % leaf_size_list[i],
    )
    plt.plot(n_estimator_range, RFC_critical_accuracy[i], ".k")


plt.xlabel("$N_\mathrm{estimators}$")
plt.ylabel("Accuracy")
lgd = plt.legend(bbox_to_anchor=(1.05, 1), loc=2, borderaxespad=0.0)

plt.show()

\end{minted}

\twocolumn


\end{document}
